\documentclass[../main.tex]{subfiles}

\begin{document}

In this project, we have developed a \gls{cryoem} image algorithm that incorporates vector compression techniques to enhance the analysis of  \gls{cryoem} images. Our research reveals that the algorithm yields excellent results for low-resolution targets, demonstrating its potential for improving throughput in \gls{cryoem} image processing. In spite of this, we observed a decrease in accuracy when applying the algorithm to higher-resolution targets.

The implementation of vector compression allowed for a significant reduction in data storage requirements with little accuracy compromises overall. The compressed vectors retained sufficient information and enabled efficient processing, making them well-suited for low-resolution alignments. This reduction in data size facilitated faster computations and enabled to store larger databases in memory.

Moreover, we have observed that the algorithm is able to resolve 3D heterogeneity even in difficult cases. Similarly, the alignment consensus is able to discard particles with unsure alignments from reconstruction. This allows to keep only images that we are sure about and provide the next steps with high quality alignment estimates.

Despite the positive outcomes observed for low-resolution targets, we observed some limitations in accuracy when requiring higher-resolution targets. The compression algorithm could not reliably retain high-frequency components of the images. Consequently, the results exhibited a plateau in which the accuracy does not increase regardless of the resolution limit used.

To address this challenge and improve the accuracy for higher-resolution targets, further research and development is required. Potential avenues for future exploration include the replacement of the Wiener filter, more effective compression algorithms and the use of weighted comparisons. These solutions will be described in depth in the \hyperlink{chap:future}{future work chapter}.

Nevertheless, the resolution range limitation does not limit the usage of the algorithm. Typical refinements involve iterative improving the current volume. This algorithm allows to accelerate the first few iterations of this cycle, which do not involve high resolution limits. Indeed, these first iterations are usually quite expensive since they must be global alignments. Thanks to the alignment consensus, the posterior local alignments will be provided with a very good starting point, reducing the odds of it falling into local minimas.

As a consequence, the presented algorithm and its results hold promise for advancing \gls{cryoem} and \gls{spa} fields. The successful application of vector compression in image alignment, particularly for low-resolution targets, opens up opportunities for more efficient and faster processing of \gls{cryoem} acquisitions.

\end{document}