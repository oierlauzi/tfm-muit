\documentclass[../main.tex]{subfiles}

\begin{document}

TO BE COMPLETED

\subsection{Influence of the Wiener CTF correction}
As mentioned in Chapter \ref{chap:implementation}, the approach to tackle the \gls{ctf} of the experimental images consists in correcting them with a Wiener filter. The issue of this approach is that not all frequencies can be recovered due to bad \gls{snr} or zero gain at the \gls{ctf}. Therefore, comparing those frequencies with the reference image may induce an artificial error. The aim of this section is to asses how this phenomenon affects the alignment accuracy.

To do so, several experiments will be carried out. Firstly, simulated images will be aligned with no \gls{ctf} being applied to them (only noise). Obviously, experimental images can not be evaluated without \gls{ctf}, as this is an artefact of the microscope. In any case, this experiment will be useful to observe the accuracy loss that can be attributed to the presence of the \gls{ctf}. 

Moreover, the ground truth alignment parameters of the simulated images is known. Therefore, once the \gls{ctf} has been applied to them, a reconstruction with these images is attempted. This gives an insight on the maximum achievable resolution with the simulated set of images. 

In the second trial, images will be clustered by their \glspl{ctf}, so that the \gls{ctf} can be assumed to be constant across all images of a given group. Then, each image group can be aligned against a reference set filtered with the representative \gls{ctf} of that group. This approach, similar to the one followed by current refinement packages, will establish the baseline for the alignment accuracy comparison. Lastly, the \gls{ctf} will be corrected with a Wiener filter, and then these images will be aligned against a clean set of reference images. 

Earlier, it was stated that most of the alignment information is contained below the resolution of $8\si{\angstrom}$. However, this alignment method targets the initial cycles of the refinement loop, where the reference volume has much less resolution. Therefore, these experiments will be carried out with a resolution limit of $15\si{\angstrom}$, so that the algorithm is evaluated on its operational range. At this resolution, typical \glspl{ctf} have one or two zero crossings. To ensure that the alignment errors can be attributed to the usage of the Wiener filter, no vector compression techniques will be used in these tests.

\subsection{Influence of the vector compression}

\subsection{Influence of the cutoff frequency}

\subsection{Combined influence of considered factors}


\end{document}
