\documentclass[../main.tex]{subfiles}

\begin{document}

TO BE COMPLETED

\subsection{Influence of the Wiener CTF correction}
As mentioned in Chapter \ref{chap:implementation}, the approach to tackle the \gls{ctf} of the experimental images consists in correcting them with a Wiener filter. The issue of this approach is that not all frequencies can be recovered due to bad \gls{snr} or zero gain at the \gls{ctf}. Therefore, comparing those frequencies with the reference image may induce an artificial error. The aim of this section is to asses how this phenomenon affects the alignment accuracy.

To do so, several experiments will be carried out. Firstly, simulated images will be aligned with no \gls{ctf} being applied to them (only noise). Obviously, experimental images can not be evaluated without \gls{ctf}, as this is an artefact of the microscope. In any case, this experiment will be useful to observe the accuracy loss that can be attributed to the presence of the \gls{ctf}. In the second trial, the experimental images will be clustered by their \glspl{ctf}, so that the \gls{ctf} can be assumed to be constant across all images of a given group. Then, each experimental image group can be aligned against a reference set filtered with the representative \gls{ctf} of that group. This approach, similar to the one followed by current refinement packages, will establish the baseline for the comparison.  In the last experiment, the experimental images' \gls{ctf} will be corrected with a Wiener filter, and then aligned against a clean set of reference images. 

All of the alignments were performed with a resolution limit of $8 \si{\angstrom}$.

As shown in the Figure \ref{}, this approach poses a slight accuracy penalty. However, it greatly simplifies the handling of the \gls{ctf} and it enables many optimisations down the road. We consider that these benefits leverage the small accuracy cost. As a consecuence, the rest of the tests will be done 

\subsection{Influence of the cutoff frequency}

\subsection{Influence of the vector compression}

\subsection{Combined influence of considered factors}




\end{document}
