\documentclass[../main.tex]{subfiles}

\begin{document}

La Microscopía Electrónica Criogénica (CryoEM) ha revolucionado el campo de la biología estructural al permitir la visualización de estructuras macromoleculares en resoluciones sin precedentes. Sin embargo, el alineamiento de imágenes sigue siendo un desafío computacional en el procesamiento de imágenes de CryoEM, ya que se utiliza en numerosos pasos. El problema de alineamiento implica encontrar la traslación y rotación óptima de una imagen para que sea lo más similar posible a otra imagen dentro de un conjunto de referencias. Por consecuencia, la gran cantidad de comparaciones requeridas para resolver el problema lo convierte en un computacionalmente costoso, haciendo que se invierta una cantidad significativa del tiempo en este proceso. Asimismo, la calidad de los resultados finales está muy influenciado por la precisión de los alineamientos.

Este proyecto introduce un nuevo método de alineamiento con el objetivo de acelerar este proceso. Para ello, se utilizarán novedosas técnicas de compresión de vectores a la hora de almacenar y comparar imágenes. Estas técnicas permiten comparar imágenes de forma eficiente, facilitando la obtención de los parámetros óptimos de alineamiento.

Otro enfoque novedoso de este trabajo es el uso del consenso de alineamiento, donde múltiples ejecuciones no deterministas se combinan para mejorar la precisión del resultado. Esto mejora la fiabilidad de los alineamientos locales posteriores, ya que estas están muy sesgadas por la solución inicial.

Se han llevado a cabo pruebas con una amplia variedad de proteínas y parámetros, obteniendo así resultados empíricamente sólidos. Estos resultados demuestran que el algoritmo tiene capacidad de mejorar el rendimiento en el procesamiento de CryoEM mientras que se mantienen niveles de precisión aceptables, particularmente para alineamientos de baja resolución. Sin embargo, se han identificado limitaciones al aplicar el algoritmo en condiciones de alta resolución.

No obstante, el algoritmo proporciona una forma rápida y precisa de resolver las primeras iteraciones de un proceso de refinamiento, que normalmente implican costosos alineamientos globales. Por lo tanto, la eficacia del algoritmo puede contribuir significativamente a aumentar el rendimiento de estas primeras iteraciones, permitiendo a los investigadores obtener resultados mucho más rápidamente. Además, gracias al consenso, las iteraciones locales posteriores serán provistas con parámetros iniciales de alta calidad, disminuyendo la posibilidad de que estas caigan en mínimos locales y, por lo tanto, mejorando los resultados finales.

\end{document}