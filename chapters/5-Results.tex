\documentclass[../main.tex]{subfiles}

\begin{document}
Throughout the development of the image alignment algorithm, some compromises were made to optimise alignment times. The purpose of this Chapter is to assess the impact of these trade-offs on accuracy, comparing them against the performance benefits obtained. The alignment tests were carried out with the 3D alignments, as this was the main focus of the project. Nevertheless, results can be extrapolated to other \gls{cryoem} problems where image alignment is used as a basis.

For this evaluation, we will utilise multiple datasets, including both simulated datasets where all parameters are controlled, as well as real datasets. By incorporating real data into the analysis, we can gain insights into the algorithm's usefulness.

\section{Test datasets}
\subfile{5.1-Test datasets}

\section{Alignment performance}
\subfile{5.2-Alignment performance}

\end{document}