\documentclass[../main.tex]{subfiles}

\begin{document}

\gls{cryoem} has revolutionised the field of structural biology by enabling the visualisation of macromolecular structures such as proteins at unprecedented resolutions. Image alignment is a core problem in \gls{cryoem}, as it is utilised by numerous steps of a typical \gls{cryoem} image processing workflow. 

The alignment problem involves finding the optimal translational and rotational transformations that align an image to a set of reference images. Thus, the vast amount of image comparisons required to solve the problem renders it a computationally expensive process. For this reason, a significant amount of time spent in \gls{cryoem} image processing is dedicated to image alignment. Similarly, the quality of the final results is heavily influenced by the accuracy of the alignments.

This project introduces a new alignment method aiming perform alignments much faster than state-of-the-art techniques at little to no accuracy degradation. To do so, novel vector compression techniques will be used when storing and comparing images. These techniques can be used to efficiently compute similarity measures between the images, facilitating the identification of the optimal alignment parameters. Moreover they require less memory footprint, allowing to store more images in memory.

Extensive experimental evaluations were conducted on publicly available datasets to validate the performance of the proposed algorithm. Comparative analysis against existing state-of-the-art methods demonstrated that the algorithm achieved comparable accuracy while significantly reducing computational time and memory usage.

\end{document}