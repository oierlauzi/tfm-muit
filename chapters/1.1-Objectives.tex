\documentclass[../main.tex]{subfiles}

\graphicspath{{\subfix{../figures/}}}

\begin{document}
The main objective of this thesis is to develop a fast and accurate image alignment algorithm for \gls{cryoem}. To be more precise, the algorithm will focus on performing 3D alignments of particles, this is, it will be used to deduce their orientation. Nevertheless, it will leave room for its application in other image alignment problems in \gls{cryoem}.

In general, the image alignment problem involves finding the best match for an image across a large set of images, also considering their rotations and shifts. As a consequence, this process involves comparing many image pairs, making it computationally expensive. In addition, it is a recurrent problem in the context of \gls{spa} and other \gls{cryoem} image processing techniques.

As a result, current \gls{spa} image processing workflows allocate significant time and resources to execute alignment algorithms. By reducing the time required for image alignment, the overall computation time needed to solve a protein structure is greatly diminished. This enhancement in performance brings forth several advantages. Firstly, it enables more efficient utilisation of the available computational resources, which are often limited due to the expenses associated with high-end workstations and servers. Additionally, it allows for higher throughput, facilitating to reach further results streaming. Last but not least, biologist are able to draw conclusions and iterate must faster, accelerating the development of drugs and vaccines.

Similarly, this widespread usage of alignment processes in \gls{cryoem} also implies that the
quality of the final results is heavily influenced by their accuracy. Therefore, it is important that the compromises taken to enhance the performance of the algorithm should not negatively impact its accuracy.

\end{document}
 
