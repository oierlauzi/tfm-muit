\documentclass[../main.tex]{subfiles}

\begin{document}

\subsection{Social impacts}
The social impacts of this development are primarily related to science. Research groups may be influenced by this project in the ways stated hereafter.

As mentioned earlier, this algorithm is used in \gls{cryoem} image processing, which is a tool used in the research process of new drugs and vaccines. Therefore, enhancing the throughput of the algorithm can reduce the development time of drugs and vaccines. When these medicines reach the market, they improve on the quality of life and longevity of humans.

Additionally, this development, in the same way as the rest of the Scipion and Xmipp framework, is \gls{foss}. This means that anyone can copy, modify and redistribute the source code. As emphasised by the lemma ``Open software accelerates science'', this directly helps other developers to improve on the state of the art, as they can continue to work on top of an existing and accessible scaffolding.

\subsection{Environmental impacts}
One of the key focuses of this algorithm is the performance improvement related to the image alignment process. Image alignment is a core problem in \gls{cryoem} image processing, so a lot of computation time is spent on running this kind of algorithms. Consequently, any performance improvements lead to a more efficient use of computational resources, drastically decreasing the electrical power consumption.

\subsection{Economic impacts}
As a consequence of the previous points, this project poses an economic impact for research facilities. Firstly, the fact that this project is being developed as a \gls{foss} implies that its usage is free of charge. Secondly, the reduction in power consumption has a direct effect on the power bill. Therefore, this two facts will reduce the operational costs of some research groups.

\end{document}